%%%%%%%%%%%%%%%%%%%%%%%%%%%%%%%%%%%%%%%
% Deedy - One Page Two Column Resume
% LaTeX Template
% Version 1.1 (30/4/2014)
%
% Original author:
% Debarghya Das (http://debarghyadas.com)
%
% Original repository:
% https://github.com/deedydas/Deedy-Resume
%
% IMPORTANT: THIS TEMPLATE NEEDS TO BE COMPILED WITH XeLaTeX
%%%%%%%%%%%%%%%%%%%%%%%%%%%%%%%%%%%%%%%
%
% Modified by Daniel Iván Parra Verde (2024)
% Modifications:
% - Used template for personal resume
% - Spanish language version
%%%%%%%%%%%%%%%%%%%%%%%%%%%%%%%%%%%%%%%

\documentclass[]{tex/deedy-resume-openfont}
\usepackage{multicol}
\usepackage{comment}

\begin{document}
\lastupdated

%%%%%%%%%%%%%%%%%%%%%%%%%%%%%%%%%%%%%%
%
%     TITLE NAME
%
%%%%%%%%%%%%%%%%%%%%%%%%%%%%%%%%%%%%%%

\namesection{Daniel Iván}{Parra Verde}{ 
Tel. +52 8447836082 | \href{mailto:danivpv@outlook.com}{Mail: danivpv@outlook.com} | Address: P.C. 25297, Saltillo, Coahuila, México\\
\href{https://github.com/danivpv}{Github: danivpv}  | \href{https://danivpv.com/}{Portfolio: danivpv.com} | \href{https://www.linkedin.com/in/danivpv/}{LinkedIn: danivpv} | \href{https://huggingface.co/danivpv}{Hugging Face: danivpv} 
}

%%%%%%%%%%%%%%%%%%%%%%%%%%%%%%%%%%%%%%
%
%     COLUMN ONE
%
%%%%%%%%%%%%%%%%%%%%%%%%%%%%%%%%%%%%%%

\begin{minipage}[t]{0.32\textwidth} 

%%%%%%%%%%%%%%%%%%%%%%%%%%%%%%%%%%%%%%
%     EDUCATION
%%%%%%%%%%%%%%%%%%%%%%%%%%%%%%%%%%%%%%

\section{Educación} 

\subsection{\href{http://www.fata.unam.mx/}{UNAM-CFATA}}
\descript{2021 | \href{https://www.enesjuriquilla.unam.mx/?page_id=4107}{Lic. Tecnología}}
\sectionsep

\subsection{\href{https://www.rug.nl/research/bernoulli/}{Univ. de Groningen (G)}}
\descript{2019 | \href{https://www.rug.nl/bachelors/artificial-intelligence/}{Lic. IA}}
\location{Países Bajos; Beca Completa}
\sectionsep

\subsection{\href{https://www.iimas.unam.mx/}{UNAM-IIMAS}}
\descript{2022 | \href{https://www.posgrado.unam.mx/matematicas/es/inicio}{MSc. Matemáticas}}
\sectionsep


%%%%%%%%%%%%%%%%%%%%%%%%%%%%%%%%%%%%%%
%     SKILLS
%%%%%%%%%%%%%%%%%%%%%%%%%%%%%%%%%%%%%%

\section{Habilidades}


\location{Lenguajes de Programación:}
Python \textbullet{} 
TypeScript \textbullet{} 
SQL \textbullet{} 
Julia \textbullet{} 
Wolfram \textbullet{} 
Bash \\[4pt]

\location{Frameworks:}
Django \textbullet{} 
FastAPI \textbullet{} 
Next.js \textbullet{} 
React \textbullet{} 
Airflow \textbullet{} 
React Native \textbullet{} 
Prefect \\[4pt]


\location{Librerías:}
Langchain \textbullet{} 
LlamaIndex \textbullet{} 
OpenCV \textbullet{} 
Pandas \textbullet{} 
PyTorch \textbullet{} 
Scikit Learn \textbullet{} 
Numpy \textbullet{} 
SQLAlchemy \textbullet{} 
Docling \textbullet{} 
Pytest \\[4pt]

\location{Servicios en la Nube:}
Hugging Face \textbullet{} 
AWS SageMaker \textbullet{} 
AWS S3 \textbullet{} 
AWS Lambda \textbullet{} 
AWS API Gateway \textbullet{} 
CometML \textbullet{} 
ZenML \textbullet{} 
Railway \textbullet{} 
Vercel \textbullet{} 
Stripe \textbullet{} 
Sendgrid \\[4pt]

\location{Bases de Datos:}
PostgreSQL \textbullet{} 
MongoDB \textbullet{} 
Qdrant \textbullet{} 
Redis \textbullet{} 
SQLite \\[4pt]

\location{Herramientas de Software:}
Docker \textbullet{} 
Git \textbullet{} 
VSCode \textbullet{} 
Cursor \textbullet{} 
\LaTeX{} \textbullet{} 
Microsoft Office \textbullet{} 
Slack \textbullet{} 
Notion \\[6pt]

\location{Idiomas:}
Inglés ($+$C1) \textbullet{} Alemán ($+$B1) \textbullet{} Español

%%%%%%%%%%%%%%%%%%%%%%%%%%%%%%%%%%%%%%
%     CURSOS
%%%%%%%%%%%%%%%%%%%%%%%%%%%%%%%%%%%%%%

\section{Cursos}
\location{Licenciatura}
Deep Learning en \custombold{Python}(G)\\
Bases de Datos Relacionales en \custombold{SQL} (G)
Inteligencia Artificial en \custombold{Python} (G)\\
Alg. y Estr. de Datos en \custombold{C} (G)\\
Álgebra Lineal\\
Series de Ecuaciones Diferenciales\\

\sectionsep

\location{Maestría}
Inferencia Estadística\\
Aprendizaje Estadístico\\
Teoría de la Información\\
Privacidad Estadística\\
Análisis numérico \custombold{Python, Julia}\\
Análisis real y funcional\\

%%%%%%%%%%%%%%%%%%%%%%%%%%%%%%%%%%%%%%
%
%     COLUMN TWO
%
%%%%%%%%%%%%%%%%%%%%%%%%%%%%%%%%%%%%%%

\end{minipage} 
\hfill
\begin{minipage}[t]{0.66\textwidth} 

%%%%%%%%%%%%%%%%%%%%%%%%%%%%%%%%%%%%%%
%     EXPERIENCE
%%%%%%%%%%%%%%%%%%%%%%%%%%%%%%%%%%%%%%

\section{Experiencia Laboral}

\runsubsection{Fundador \& Ingeniero de IA - \href{https://mxai.dev}{MXAI}}\\
\location{Enero 2024 – Presente | Remoto}
\vspace{\topsep}
\begin{tightemize}
    \item Desarrollo y despliegue de soluciones basadas en IA usando \custombold{AWS Lambda}, \custombold{Redis}, y arquitecturas dirigidas por eventos, similar a \custombold{Azure Functions}, para backends de chatbots con \custombold{Langchain} y \custombold{RAG}.
    \item Desarrollo de sitio web corporativo usando stack moderno (\custombold{Next.js}, \custombold{TypeScript}, \custombold{Tailwind}, \custombold{shadcn/ui}) con diseño responsivo.
    \item Liderazgo de reuniones de ventas con clientes y presentaciones de demostración, mejorando el compromiso y satisfacción del cliente.
\end{tightemize}
\sectionsep

\runsubsection{Científico de Datos - \href{https://kuona.ai/es/}{Kuona}}\\
\location{Diciembre 2023 – Mayo 2024 | Remoto}
\begin{tightemize}
    \item Desarrollo de \custombold{RNNs} con \custombold{PyTorch} y servido a través de un sistema ML por lotes usando \custombold{Django} ORMs y \custombold{PostgreSQL} para cliente y equipo de operaciones para simular escenarios de gestión de ingresos.
    \item Propuesta de uso de \custombold{QLora} para fine-tuning de modelos y funciones de pérdida personalizadas para mejorar KPIs de negocio.
    \item Modelado de datos de precios de competidores e implementación de LIME para análisis de elasticidad de demanda usando \custombold{Pandas}, \custombold{Pytorch}, \custombold{Scikit-Learn} y \custombold{PostgreSQL}.
\end{tightemize}
\sectionsep

\runsubsection{Científico de Datos Jr. - \href{https://www.linkedin.com/company/entropia-ai/}{Entropía AI}}\\
\location{Febrero 2023 – Noviembre 2023 | Remoto}
\begin{tightemize}
    \item Liderazgo en el desarrollo de pipelines \custombold{RAG}, mejorando la precisión en benchmarks de QA para datasets legales y académicos con \custombold{LlamaIndex}, \custombold{MongoDB} y \custombold{LangChain}.
    \item Creación de pipeline ELT con \custombold{Python}, \custombold{bs4} y \custombold{Airflow} para datos del Sistema Político Mexicano en \custombold{MongoDB}.
    \item Contribución al desarrollo ágil de la aplicación web \href{http://www.icaro.ai/home/}{Icaro AI} alojada en \custombold{Heroku}.
\end{tightemize}
\sectionsep

\runsubsection{Desarrollador Interno - \href{https://www.wolfram.com/company/}{Wolfram Research}}\\
\location{Febrero 2021 – Febrero 2022 | Remoto}
\begin{tightemize}
    \item Desarrollo de la aplicación \href{https://education.wolfram.com/summer-camp/}{'Wolfram Summer Camp'} usando \href{https://www.wolfram.com/wolfram-one/}{\custombold{Wolfram One}}.
    \item Rotación al equipo de desarrollo de \href{https://www.wolframalpha.com/}{Wolfram Alpha}.
\end{tightemize}
\sectionsep


\section{Proyectos}

\runsubsection{ArXiv Domain Expert LLM}\\
\location{\href{https://github.com/danivpv/LLM-ArXiv-Domain-Expert}{GitHub Repository}}
\begin{tightemize}
    \item Desarrollo de sistema de ML para finetunning de LLM desde creación de datasets en \custombold{Hugging Face} hasta entrenado y desplegado en \custombold{AWS SageMaker}.
    \item Utiliza \custombold{MongoDB}, \custombold{Qdrant}, \custombold{ZenML}, \custombold{CometML} y \custombold{AWS} para infraestructura.
\end{tightemize}
\sectionsep

\runsubsection{Tesis Maestría: Identificación de modelo RNN con EKF}\\
\location{\href{https://www.linkedin.com/in/esteban-a-hernandez-vargas-04724321/}{PhD. Esteban A. Hernandez-Vargas} | \href{https://www.systemsmedicine.de/Research/}{Systems Medicine of Infectious Diseases}}
\begin{tightemize}
    \item Implementación de algoritmo de identificación de modelo para redes neuronales recurrentes en \custombold{Julia} usando Teoría de Filtro de Kalman Extendido.
    \item Análisis de error de optimización, aproximación y generalización estadística.
\end{tightemize}
\sectionsep

\runsubsection{"Convocatoria Jóvenes Talentos 2018" CONCyTEC}\\
\location{\href{mailto:rchavez@comunidad.unam.mx}{PhD. Rafael Chávez Moreno} | \href{https://www.ingenieria.unam.mx/uat/directorio.php}{Unidad de Alta Tecnología Juriquilla, Qro.}}
\begin{tightemize}
    \item Obtención de fondos ($\$89,000$ MXN) para diseño y manufactura de Rover.
    \item Responsable de los subsistemas de visión por computadora y planeación de trayectoria usando Raspberry Pi y sensor Kinect con \custombold{Python}.
\end{tightemize}


\end{minipage} 
\end{document}  