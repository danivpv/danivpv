%%%%%%%%%%%%%%%%%%%%%%%%%%%%%%%%%%%%%%%
% Deedy - One Page Two Column Resume
% LaTeX Template
% Version 1.1 (30/4/2014)
%
% Original author:
% Debarghya Das (http://debarghyadas.com)
%
% Original repository:
% https://github.com/deedydas/Deedy-Resume
%
% IMPORTANT: THIS TEMPLATE NEEDS TO BE COMPILED WITH XeLaTeX
%%%%%%%%%%%%%%%%%%%%%%%%%%%%%%%%%%%%%%%

\documentclass[]{tex/deedy-resume-openfont}
\usepackage{multicol}
\usepackage{comment}

\begin{document}
\lastupdated

%%%%%%%%%%%%%%%%%%%%%%%%%%%%%%%%%%%%%%
%
%     TITLE NAME
%
%%%%%%%%%%%%%%%%%%%%%%%%%%%%%%%%%%%%%%

\namesection{Daniel Iván}{Parra Verde}{ 
Tel. +52 5644550582 | Dirección: P.C. 25297, Saltillo, Coahuila, México\\
\href{mailto:danivpv@gmail.com}{Mail: danivpv@outlook.com} | 
\href{https://www.instagram.com/daniel.ivan.v/}{Instagram: daniel.ivan.v} \\ \href{https://www.linkedin.com/in/danielivanparraverde/}{LinkedIn: danielivanparraverde} | \href{https://github.com/danivpv}{GitHub: danivpv}}

%%%%%%%%%%%%%%%%%%%%%%%%%%%%%%%%%%%%%%
%
%     COLUMN ONE
%
%%%%%%%%%%%%%%%%%%%%%%%%%%%%%%%%%%%%%%

\begin{minipage}[t]{0.33\textwidth} 

%%%%%%%%%%%%%%%%%%%%%%%%%%%%%%%%%%%%%%
%     EDUCATION
%%%%%%%%%%%%%%%%%%%%%%%%%%%%%%%%%%%%%%

\section{Educación} 

\subsection{\href{http://www.fata.unam.mx/}{UNAM-CFATA}}
\descript{Licenciatura en Tecnología}
\location{2021 | Juriquilla, Qro.}
\sectionsep

\subsection{\href{https://www.rug.nl/research/bernoulli/}{Universidad de\\Groningen (G)}}
\descript{Semestre de Intercambio: Beca}
\location{2019 | Groningen, Países Bajos}
\sectionsep

\subsection{\href{https://www.posgrado.unam.mx/matematicas/es/inicio}{UNAM-IIMAS}}
\descript{MSc. Ciencias Matemáticas}
\location{2022 | Juriquilla, Qro.}
\sectionsep

%%%%%%%%%%%%%%%%%%%%%%%%%%%%%%%%%%%%%%
%     COURSEWORK
%%%%%%%%%%%%%%%%%%%%%%%%%%%%%%%%%%%%%%

\section{Cursos Relevantes}
\subsection{Licenciatura}
Inteligencia Artificial en Python (G)\\
Aprendizaje Profundo en Python (G)\\
Algoritmos y Estructuras de \\
Datos en C (G)\\
Bases de Datos PostgreSQL (G)\\
Computo Científico\\
Teoría de Grafos\\
Serie de Ecuaciones Diferenciales\\
Dinámica No-linear\\
Dinámica Evolutiva\\
Mecánica Estadística\\
Algebra Linear
\sectionsep

\subsection{Maestría}
Aprendizaje Máquina Teórico\\
Teoría de la Información\\
Inferencia Estadística\\
Fundamentos Estadísticos de la Privacidad\\
Análisis Numérico\\
Análisis Real\\
Análisis Funcional\\
NN Biológicas y Artificiales\\


%%%%%%%%%%%%%%%%%%%%%%%%%%%%%%%%%%%%%%
%     SKILLS
%%%%%%%%%%%%%%%%%%%%%%%%%%%%%%%%%%%%%%

\section{Aptitudes}
\subsection{Programación}
\location{Competente en:}
Python \textbullet{} Julia \textbullet{} Wolfram \textbullet{} Bash \\ 
\location{Familiar}
C \textbullet{} Matlab  \textbullet{} CUDA \textbullet{} R 
\sectionsep

\subsection{Software}
\LaTeX\ \textbullet{} Microsoft Office \textbullet{} VSCode \textbullet{} Git
\sectionsep

\subsection{Lenguajes}
Ingles ($+$C1) \textbullet{} Alemán ($+$B1) \textbullet{} Español

%%%%%%%%%%%%%%%%%%%%%%%%%%%%%%%%%%%%%%
%
%     COLUMN TWO
%
%%%%%%%%%%%%%%%%%%%%%%%%%%%%%%%%%%%%%%

\end{minipage} 
\hfill
\begin{minipage}[t]{0.66\textwidth} 

%%%%%%%%%%%%%%%%%%%%%%%%%%%%%%%%%%%%%%
%     EXPERIENCE
%%%%%%%%%%%%%%%%%%%%%%%%%%%%%%%%%%%%%%

\section{Experiencia Laboral}

\runsubsection{Científico de Datos - \href{https://www.linkedin.com/company/entropia-ai/}{Entropía AI}}\\
\location{Febrero 2023 – Ahora | Remoto}
\vspace{\topsep}
\vspace{\topsep} % Hacky fix for awkward extra vertical space
\begin{tightemize}
\item Responsable de tareas de procesamiento de lenguaje natural (NLP) como ingeniería de prompts, sumarización y filtrado de documentos por similitud semántica con la API de \href{https://openai.com/}{OpenAI} de grandes modelos de lenguaje (LLM). Empecé la certificación de \href{https://learn.microsoft.com/en-us/certifications/azure-ai-engineer/}{Ingeniero Asociado de Inteligencia Artificial de Microsoft Azure}.
\item Creación de una pipeline con Python de ingestión, transformación y carga de datos en los servicios de Microsoft Azure Synapse. Los datos incluían tablas y textos legislativos y regulatorios del sistema político mexicano. Empecé la certificación de \href{https://learn.microsoft.com/en-us/certifications/azure-data-engineer/}{Ingeniero de Datos de Microsoft Azure}.
\item Colaboración en la implementación de interfaces web para varias aplicaciones en Python con Dash y Flask.
\item Participación en reuniones con clientes para la integración ágil de retroalimentación.
\end{tightemize}
\sectionsep

\runsubsection{Contratista Independiente \href{https://www.wolfram.com/company/}{Wolfram Research}}\\
\location{Febrero 2021 – Febrero 2023 | Remoto}
\vspace{\topsep} % Hacky fix for awkward extra vertical space
\begin{tightemize}
\item Implementación y administración de sitio web para el evento anual de \href{https://education.wolfram.com/summer-camp/}{'Wolfram Summer Camp'} usando las tecnologías en la nube de Wolfram One para automatizar reportes y manejar recursos.
\item Rotación al equipo de desarrollo del software de \href{https://www.wolframalpha.com/}{Wolfram Alpha} donde empecé a conocer el ciclo de desarrollo de un producto de software maduro.
\end{tightemize}
\sectionsep


\section{Proyectos Académicos}

\runsubsection{Tesis Maestría: Identificación de Modelo de Red Neuronal Recurrente de Orden Superior con Teoría de Filtro de Kalman}
\descript{\\\href{https://www.linkedin.com/in/esteban-a-hernandez-vargas-04724321/}{PhD. Esteban A. Hernandez-Vargas}}
\location{Abril 2022 – Enero 2023 | \href{https://www.systemsmedicine.de/Research/}{Systems Medicine of Infectious Diseases}}
\vspace{\topsep} % Hacky fix for awkward extra vertical space
\begin{tightemize}
\item Implementación de algoritmo de identificación tambien conocido como entrenamiento de una red neuronal recurrente con el lenguaje de Julialang y la teoría del Filtro de Kalman Extendido.
\item Análisis del error de estimación, optimización, aproximación y generalización.
\item Simulaciones empíricas de infecciones virales dentro de huésped de COVID-19 entre otras enfermedades.
\end{tightemize}
\sectionsep

\runsubsection{"Convocatoria Jóvenes Talentos 2018" CONCyTEC}
\descript{\\\href{mailto:rchavez@comunidad.unam.mx}{PhD. Rafael Chávez Moreno}}
\location{Febrero 2018 – Octubre 2018 | \href{https://www.ingenieria.unam.mx/uat/directorio.php}{Unidad de Alta Tecnología Juriquilla, Qro.}}
\vspace{\topsep} % Hacky fix for awkward extra vertical space
\begin{tightemize}
\item Obtención de fondo económico ($\$89,000$ MXN) para el diseño y manufactura de un robot tipo Rover usando componentes de bajo costo para su presentación a audiencias técnicas y no técnicas.
\item Responsable del subsistema de visión y planeación de trayectoria donde aprendí sobre el diseño e importancia de requerimientos.
\item Programación en Python de Raspberry Pi con sensor Kinect de para planeación de trayectoria y procesamiento de imágenes con OpenCV, Numpy, sci-kit learn.
\end{tightemize}

%%%%%%%%%%%%%%%%%%%%%%%%%%%%%%%%%%%%%%
%     RESEARCH
%%%%%%%%%%%%%%%%%%%%%%%%%%%%%%%%%%%%%%
\begin{comment}
\section{Research}
\runsubsection{Data Driven Inference of Complex Dynamics}
\descript{\\Dr. Jorge X. Velasco}
\location{Oct 2019 –- | Instituto de Matemáticas UNAM unidad Juriquilla, QRO.}
\vspace{\topsep} % Hacky fix for awkward extra vertical space
\begin{tightemize}
\item Analysis and caracterization of Dengue incidence cases in Mexico database.
\item Study of Koopman operator theory for non-linear dynamical system analysis including state of the art DMD algorithms for data-driven inference.
\item Hankel alternative view of Koopman (HAVOK) implementation in Julia with automatic estimation of suitable parameter space. DOI:10.1038/s41467-017-00030-8
\end{tightemize}
\sectionsep

\end{comment}
%%%%%%%%%%%%%%%%%%%%%%%%%%%%%%%%%%%%%%
%     Extracurriculars
%%%%%%%%%%%%%%%%%%%%%%%%%%%%%%%%%%%%%%
\begin{comment}
    
\section{Extracurriculares} 
\begin{tabular}{rll}
2018/05 & \href{https://www.matcuer.unam.mx/actividades/EscuelaVerano.php}{Instituto de Matemáticas UNAM XV Escuela de Verano en Matemáticas}\\
2018/06 & \href{https://sites.google.com/view/dinamicapotosina2021/}{Instituto de Física UASLP Escuela de Verano en Sistemas Dinámicos}\\
2018/10 & (Winner) \href{https://www.linkedin.com/pulse/geiq-hackathon-challenging-young-minds-michael-tresca/}{GEIQ HACKATHON} Qro. Qro.\\
2018/10 & \href{https://dgapa.unam.mx/index.php/impulso-a-la-investigacion/papiit}{PAPIIT Beca de Titulación} \\
2019/10 & \href{https://hackafest.sparkassenstiftung-latinoamerica.org
}{Hackafest Sparkassenstiftung en Puebla}\\
2019/10 & \href{https://ece2019.eventos.cimat.mx/}{Escuela de Cómputo Evolutivo CIMAT}
\end{tabular}

\end{comment}

\end{minipage} 
\end{document}  