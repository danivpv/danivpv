%%%%%%%%%%%%%%%%%%%%%%%%%%%%%%%%%%%%%%%
% Deedy - One Page Two Column Resume
% LaTeX Template
% Version 1.1 (30/4/2014)
%
% Original author:
% Debarghya Das (http://debarghyadas.com)
%
% Original repository:
% https://github.com/deedydas/Deedy-Resume
%
% IMPORTANT: THIS TEMPLATE NEEDS TO BE COMPILED WITH XeLaTeX
%%%%%%%%%%%%%%%%%%%%%%%%%%%%%%%%%%%%%%%

\documentclass[]{deedy-resume-openfont}
\usepackage{multicol}
\usepackage{comment}

\begin{document}
\lastupdated

%%%%%%%%%%%%%%%%%%%%%%%%%%%%%%%%%%%%%%
%
%     TITLE NAME
%
%%%%%%%%%%%%%%%%%%%%%%%%%%%%%%%%%%%%%%

\namesection{Daniel Iván}{Parra Verde}{ 
Tel. +52 8447836082 | Address: P.C. 25297, Saltillo, Coahuila, México\\
\href{mailto:danivpv@outlook.com}{Mail: danivpv@outlook.com} | \href{https://www.linkedin.com/in/danielivanparraverde/}{LinkedIn: danielivanparraverde}
}

%%%%%%%%%%%%%%%%%%%%%%%%%%%%%%%%%%%%%%
%
%     COLUMN ONE
%
%%%%%%%%%%%%%%%%%%%%%%%%%%%%%%%%%%%%%%

\begin{minipage}[t]{0.32\textwidth} 

%%%%%%%%%%%%%%%%%%%%%%%%%%%%%%%%%%%%%%
%     EDUCATION
%%%%%%%%%%%%%%%%%%%%%%%%%%%%%%%%%%%%%%

\section{Education} 

\subsection{\href{http://www.fata.unam.mx/}{UNAM-CFATA}}
\descript{2021 | \href{https://www.enesjuriquilla.unam.mx/?page_id=4107}{BSc. Technology}}
\location{Focus on technological innovation}
\sectionsep
\subsection{\href{https://www.rug.nl/research/bernoulli/}{Groningen University (G)}}
\descript{2019 | \href{https://www.rug.nl/bachelors/artificial-intelligence/}{Artificial Intelligence}}
\location{Netherlands; Full Scholarship: Semester's tuition and living expenses}
\sectionsep

\subsection{\href{https://www.iimas.unam.mx/}{UNAM-IIMAS}}
\descript{2022 | \href{https://www.posgrado.unam.mx/matematicas/es/inicio}{MSc. Mathematical Sciences}}
\location{}
\sectionsep

%%%%%%%%%%%%%%%%%%%%%%%%%%%%%%%%%%%%%%
%     COURSEWORK
%%%%%%%%%%%%%%%%%%%%%%%%%%%%%%%%%%%%%%

\section{Relevant Coursework}
\subsection{Bachelor}
Artificial Intelligence in \custombold{Python} (G)\\
Deep Learning in \custombold{Python}(G)\\
Algorithms and Data Structures in \custombold{C} (G)\\
Relational Databases in \custombold{PostgreSQL} (G)
Scientific Programming \custombold{Python, Julia}\\
Linear Algebra\\
Graph Theory \custombold{Python}\\
Non-linear Dynamical Systems \custombold{Wolfram}\\
Mechanical Statistics \custombold{Wolfram}\\
Differential Equations Series\\
Evolutive Dynamics \custombold{Wolfram}\\

\sectionsep

\subsection{Master}
Theoretic Machine Learning\\
Information Theory\\
Statistical Inference\\
Statistical Foundations of Privacy\\
Numerical Analysis \custombold{Python, Julia}\\
Real Analysis\\
Functional Analysis\\
Biological and Artificial NN \custombold{Python}\\


%%%%%%%%%%%%%%%%%%%%%%%%%%%%%%%%%%%%%%
%     SKILLS
%%%%%%%%%%%%%%%%%%%%%%%%%%%%%%%%%%%%%%

\section{Skills}
\subsection{Programming}
\location{Competent:}
\custombold{Python} \textbullet{} Julia \textbullet{} Wolfram \textbullet{} Bash \\ 
\location{Familiar:}
C \textbullet{} Matlab  \textbullet{} CUDA \textbullet{} R 
\sectionsep

\subsection{Software}
\LaTeX\ \textbullet{} Microsoft Office \textbullet{} VS-Code \textbullet{} Git
\sectionsep

\subsection{Languages}
English ($+$C1) \textbullet{} German ($+$B1) \textbullet{} Spanish

%%%%%%%%%%%%%%%%%%%%%%%%%%%%%%%%%%%%%%
%
%     COLUMN TWO
%
%%%%%%%%%%%%%%%%%%%%%%%%%%%%%%%%%%%%%%

\end{minipage} 
\hfill
\begin{minipage}[t]{0.66\textwidth} 

%%%%%%%%%%%%%%%%%%%%%%%%%%%%%%%%%%%%%%
%     EXPERIENCE
%%%%%%%%%%%%%%%%%%%%%%%%%%%%%%%%%%%%%%

\section{Work Experience}
\runsubsection{Data Scientist}\\
\location{\href{https://kuona.ai/es/}{Kuona} | December 2023 – May 2024 | Remote}
\vspace{\topsep} % Hacky fix for awkward extra vertical space
\begin{tightemize}
\item Modeling of retailers competitors' pricing data to improve client's net sales forecast with \custombold{Python}, \custombold{Pandas} and \custombold{SQL}. 
\item Training, hyper parameter tuning and deployment of RNNs models optimizing the sales units of various manufacturers and retailers, resulting in improved forecasting accuracy with \custombold{Python} on \custombold{AWS}. 
\item Implemented interpretable AI technique LIME to compute point-wise sensitivity of parameters. In particular, demand forecast sensitivity to price and competitors prices, i.e. elasticity of demand and cross-elasticity of demand respectively. Used \custombold{Pytorch} and \custombold{Scikit-Learn}.
\item Collaboration with operations team in scenario simulations for clients' dynamic pricing strategies and revenue management.
\end{tightemize}
\sectionsep

\runsubsection{Data Scientist Jr.}\\
\location{\href{https://www.linkedin.com/company/entropia-ai/}{Entropía AI} | February 2023 – November 2023 | Remote}
\begin{tightemize}
\item Responsible for \custombold{NLP} tasks like QA and summarizing using custom state-of-the-art techniques of semantic similarity retrieval of documents, retrieval augmented generation of text and prompt engineering using \custombold{Python}, \custombold{Azure} Services and open source frameworks like \href{https://docs.llamaindex.ai/en/stable/}{\custombold{LlamaIndex}}, \href{https://www.langchain.com/}{\custombold{LangChain}}, etc...
\item Collaborated in the creation of \custombold{ELT} pipeline with \custombold{Python} and \custombold{Airflow} of heterogeneous data sources. Data included tables and texts of official legislative and regulatory information from the Mexican Political System.
\item \custombold{Agile} methodology with Kanban and Feature Driven Methodology to develop \href{http://www.icaro.ai/home/}{Icaro AI} web app using \custombold{Streamlit}'s UI, \custombold{Python} and hosted in \custombold{Heroku}.
\item Hands on role in client meetings to integrate feedback and discuss features.
\end{tightemize}
\sectionsep

\runsubsection{Developer Internship}\\
\location{\href{https://www.wolfram.com/company/}{Wolfram Research} | February 2021 – February 2022 | Remote}
\begin{tightemize}
\item Implementation and management of web app for \href{https://education.wolfram.com/summer-camp/}{'Wolfram Summer Camp'} using the cloud services of \href{https://www.wolfram.com/wolfram-one/}{\custombold{Wolfram One}}.
\item Rotation into \href{https://www.wolframalpha.com/}{Wolfram Alpha} software development team. 
% where I got introduced to DevOps tools like \custombold{Bitbucket}, \custombold{Jenkins}, testing practices and code reviews. Planned through \custombold{Waterfall model} and worked with Wolfram, Java and \custombold{Python}.
\end{tightemize}
\sectionsep


\section{Academic Projects}

\runsubsection{Master Thesis: Neural Model Identification with Extended Kalman Filter}\\
\location{\href{https://www.linkedin.com/in/esteban-a-hernandez-vargas-04724321/}{PhD. Esteban A. Hernandez-Vargas} | \href{https://www.systemsmedicine.de/Research/}{Systems Medicine of Infectious Diseases}}
\vspace{\topsep} % Hacky fix for awkward extra vertical space
\begin{tightemize}
\item Implementation of model identification algorithm for recurrent neural networks in \custombold{Julia} programming language using Extended Kalman Filter Theory.
\item Error analysis of estimation, optimization, approximation and statistical generalization. 
\item Empirical (synthetic data) simulations of within-host viral infections caused by the viruses of SARS-COV-2, HIV and Influenza-A.
\end{tightemize}
\sectionsep

\runsubsection{"Convocatoria Jóvenes Talentos 2018" CONCyTEC}\\
\location{\href{mailto:rchavez@comunidad.unam.mx}{PhD. Rafael Chávez Moreno} | \href{https://www.ingenieria.unam.mx/uat/directorio.php}{Unit of High Technology Juriquilla, Qro.}}
\vspace{\topsep} % Hacky fix for awkward extra vertical space
\begin{tightemize}
\item Secured funding ($\$89,000$ MXN) for the design and manufacture of a Rover through technical and non-technical writings and presentations.
\item Responsible for the computer vision and path planning subsystem instrumented with a Raspberry Pi and a Kinect sensor with \custombold{Python}. 
\end{tightemize}

%%%%%%%%%%%%%%%%%%%%%%%%%%%%%%%%%%%%%%
%     RESEARCH
%%%%%%%%%%%%%%%%%%%%%%%%%%%%%%%%%%%%%%
\begin{comment}
%%%%%%%%%%%%%%%%%%%%%%%%%%%%%%%%%%%%%%
%     Extracurriculars
%%%%%%%%%%%%%%%%%%%%%%%%%%%%%%%%%%%%%%
\section{Extracurriculars} 
\begin{tabular}{rll}
2018/10 & (Winner) 
\end{tabular}
\section{Research}
\runsubsection{Data Driven Inference of Complex Dynamics}
\descript{\\Dr. Jorge X. Velasco}
\location{Oct 2019 –- | Instituto de Matemáticas UNAM unidad Juriquilla, QRO.}
\vspace{\topsep} % Hacky fix for awkward extra vertical space
\begin{tightemize}
\item Analysis and caracterization of Dengue incidence cases in Mexico database.
\item Study of Koopman operator theory for non-linear dynamical system analysis including state of the art DMD algorithms for data-driven inference.
\item Hankel alternative view of Koopman (HAVOK) implementation in Julia with automatic estimation of suitable parameter space. DOI:10.1038/s41467-017-00030-8
\end{tightemize}
\sectionsep


2018/05 & \href{https://www.matcuer.unam.mx/actividades/EscuelaVerano.php}{Instituto de Matemáticas UNAM XV Escuela de Verano en Matemáticas}\\
2018/06 & \href{https://sites.google.com/view/dinamicapotosina2021/}{Instituto de Física UASLP Escuela de Verano en Sistemas Dinámicos}\\
2019/10 & \href{https://ece2019.eventos.cimat.mx/}{Escuela de Cómputo Evolutivo CIMAT}

2018/10 & \href{https://dgapa.unam.mx/index.php/impulso-a-la-investigacion/papiit}{PAPIIT Beca de Titulación} \\
2019/10 & \href{https://hackafest.sparkassenstiftung-latinoamerica.org
}{Hackafest Sparkassenstiftung en Puebla}\\
\end{comment}
\end{minipage} 
\end{document}  